\documentclass{compactslide} %% DEPEND

\author{Jae. K. Lee \\
Teela James \\
\texttt{University of Virginia}}
\title{GEOSS Public User Analysis Service}
\date{}

\begin{document}

\titleslide

\begin{slide}
\heading{Introduction}
GEOSS (Gene Expression Open Source System) is a system for the storage
and analysis of microarray data.  GEOSS provides a variety of functions:
\begin{itemize}
\item repository for gene chip data
\item used by the array lab at the University of Virginia to manage
    ordering, processing, and billing of gene chips 
\item used by investigators at the University of Virginia to retrieve,
    analyze, and share chip data
\item used by public users to access and analyze gene chip data
\end{itemize}
\end{slide}

\begin{slide}
\heading{Introduction}
This presentation illustrates the public user interface and how
public users can analyze gene chip data.  Specifically, it addresses:
\begin{itemize}
\item how to create an analysis tree that analyzes others' data
\item how to create an analysis tree to analyze your own data
\item how to view the results files
\item what each of the current analysis modules does and what their
    results mean
\end{itemize}
\end{slide}

\begin{slide}
\heading{Getting Started}

The UVa public server is at: \\
\texttt{http://biostat.virginia.edu/geoss/site}
\begin{itemize}
\item accounts may be requested via the ``Request An Account'' link
\item public user accounts are automatically generated and users may login
    immediately with their new account
\end{itemize}
\end{slide}

\begin{slide}
\heading{Member Home}
The ``Member Home'' page is the main menu for public users. There are four 
main activity areas:
\begin{itemize}
\item Array Study Management
\item File Management
\item Analysis Management
\item Account Management
\end{itemize}
\end{slide}

\begin{slide}
\heading{View All Studies}

The ``View All Studies'' link provides an overview of publicly
accessible studies.  Users interested in viewing and analyzing data sets
made available by investigators should use this link as a starting
point.
\begin{itemize}
\item searching and sorting capabilities can be used to identify studies of
  interest
\item clicking on the ``Study Name'' link will provide information on the
  source of the study  (typically the PubMed entry)
\item clicking on ``Full View'' provides detailed information about a study,
  including the study abstract, the condition grouping, the default
  GEOSS analysis results, and a shortcut to creating your own analysis
  tree using the data from the study in question
\end{itemize}
\end{slide}

\begin{slide}
\heading{File Management}
The ``View My Files'' link allows the user to view and/or download files that
they have access to.  For public users, this is primarily analysis
results files.
\begin{itemize}
\item use the ``Directories'' section to navigate to the directory you wish to
  view.  Note that results files for each analysis tree are in their own
  directory
\item use the ``Files'' section to view/download a file by clicking on the
  link in the ``File'' column
\end{itemize}
\end{slide}

\begin{slide}
\heading{Analysis Trees}

GEOSS analyses are called ``Analysis Trees''.  Each tree consists of one
or more analyses being performed on a set of input data.  In order to
create an analysis tree, we must define several items:
\begin{itemize}
\item the data source
\item the data grouping
\item the analyses to perform, including their order and parameters
\end{itemize}
\end{slide}

\begin{slide}
\heading{Data Sources}

There are three possible data sources available to public users:
\begin{enumerate}
\item Array Study \\
A study that is already defined in the system and has been made
accessible to public users.  The data is already loaded and the chips
have been grouped into conditions.  These include a collection of
publicly-available biomedical microarray data sets.
\item Analysis Set \\
A customized grouping of chips into conditions, and conditions into
sets.  The chip data may have been uploaded by the user or may be
obtained from data other users have made available.
\item UVa Public Data \\
UVa array studies that have been made publicly available.  The data is
already loaded and the chips have been grouped into conditions.
\end{enumerate}
\end{slide}

\begin{slide}
\heading{Creating a Tree}

Trees can be created using the ``Create/Run a new analysis tree'' link.
\begin{itemize}
\item provide a unique name
\item specify a structure (we recommend starting with the default
structure and modifying it as needed)
\item designate a data source type and then click ``Choose Source''
\item pick the specific source then click ``Next''
\end{itemize}
\end{slide}

\begin{slide}
\heading{Refining a tree}

The ``Edit Analysis Tree'' page allows you to refine the tree structure.
You can:
\begin{itemize}
\item Add/Remove modules
\item Set the parameters for each node
\item View help for a particular node
\end{itemize}

When you have completed tree configuration, you may run the tree by
clicking ``Run Analysis''
\end{slide}

\begin{slide}
\heading{Tree Result Files}

You will receive an email indicating the status of your analysis when
the analysis is complete.  Tree result files can be viewed via the
``View My Files'' link.  Alternatively, several shortcuts to tree
results are provided:
\begin{itemize}
\item the ``View tree files'' link from Edit Analysis Tree
\item the ``View Analysis Result Files'' from the Full View inside the Study
  Viewer
\item the ``file repository'' link from ``Run Analysis Tree''
\item a link in the status email
\end{itemize}
Results files include:
\begin{itemize}
\item module-specific output files
\item error log for each module
\end{itemize}
\end{slide}

\begin{slide}
\heading{Advanced Data Preparation}

The Advanced Data Preparation links allow users to specify their own
grouping.  This allows them to use their own data or to specify
customized grouping of others data.

You must upload your own data prior to grouping and analyzing it.  
\end{slide}

\begin{slide}
\heading{Upload external array data file or a criteria file}

To upload data use the  ``Upload external array data file or a criteria 
file'' link.  

\begin{itemize}
\item currently, only data files generated using mas5 or GCOS may be
    uploaded
\item specify the data file to upload
\item specify the chip type associated with the data file
\item provide a unique, meaningful name for the chip data
\end{itemize}

Click on ``Upload'', to upload the data.  Note that if you are uploading
large data files, there may be some delays associated with this action.

Currently public users can only load chips individually. Future versions
may make bulk chip loading available to public users.
\end{slide}

\begin{slide}
\heading{Grouping Terminology}

Analysis Condition - a grouping of several chips into one condition.
Chips in the same condition should be either biological or technical
replicates.  As such, they will have the same chip layout.  There must
be one or more chips in each condition.

Analysis Set - a grouping of analysis conditions.  There must be two
or more conditions in an analysis set.
\end{slide}

\begin{slide}
\heading{Building an Analysis Condition}

The ``Create new analysis condition'' link can be used to build an
analysis condition.

Hybridizations can be specified from a variety of sources:
\begin{itemize}
\item uploaded hybridizations are accessible from choosing
    ``Hybridizations'' as the source.
\item multiple hybridizations can be added at once by choosing one of the
    other options.  For instance, choosing studies as a source would add
    all conditions associated with a specific study to the analysis
    condition.
\item criteria files can be used to select hybridizations to add based on
    specified criteria.  For instance, your criteria file might specify
    the gender and age associated with a set of hybridizations.  You
    could then specify a condition that contains only samples taken from
    males over 50.
\end{itemize}
\end{slide}

\begin{slide}
\heading{Building an Analysis Set}

The ``Create new analysis set'' link can be used to build an analysis set.

Again, several sources are available:
\begin{itemize}
\item if you select ``Analysis Conditions'' as the source, you can choose the
  conditions you just created to be included in the set
\item other options provide shortcuts.  For instance, if you choose an
  entire study, the system will automatically build the analysis
  conditions for you and you can refine those conditions as appropriate
\end{itemize}

As discussed earlier, ``Analysis Sets'' are one of the possible inputs to
a tree.  The newly created set can now be analyzed.       
\end{slide}

\begin{slide}
\heading{Example:  Analyze your own data}
\begin{enumerate}
\item Upload each chip file using ``Upload an external array data file or
criteria file''
\item Group all the chips in to analysis conditions using ``Create new
analysis condition''
\item Group all the analysis conditions together into an analysis set using
``Create new set of conditions''
\item Click on ``Create tree using this Set''
\item Run the tree
\item View the results via ``View my files''
\end{enumerate}
\end{slide}

\begin{slide}
\heading{Example: Get analysis results from public data}
\begin{enumerate}
\item Use ``View all studies'' to find a study of interest
\item Click on ``Full View'' for detail on the study.  Use ``View Analysis
Result File'' to see the default analysis results for the study.
\item If you wish to create a customized tree, provide a name and 
click on ``Create Customized Data Set for Analysis'' from the full view.
\item Customize set and conditions
\item Click on ``Create tree using this Set''
\item Run the tree
\item View the results via ``View my files''
\end{enumerate}
\end{slide}

\begin{slide}
\heading{Account Management}

These links provide the ability to modify account settings including:
\begin{itemize}
\item personal info (email address)
\item password
\item other users in your group who may see your data
\end{itemize}
\end{slide}

\begin{slide}
\heading{Future GEOSS development}

Future GEOSS development plans include:
\begin{itemize}
\item add more public, biomedical microarray data sets based on user
feedback
\item add more analysis tools, such as classification methods (LDA, QDA,
SVM) for biomarker discovery based on user demand
\item support cDNA data and other customized microarrays
\end{itemize}
\end{slide}

\begin{slide}
\heading{Availability}

It is not necessary to download GEOSS in order to use the public
service.  However if you are interested in installing your own version
of GEOSS to support an array center processing chip data, to analyze
data, or to build custom analyses, GEOSS can be obtained from: \\
\texttt{http://sourceforge.net/projects/geoss/}

Questions can be sent to: \\
  \texttt{geoss@biostat.virginia.edu}
\end{slide}

\end{document}
